
\subsection{Segundo juego de pruebas} \label{sec:prueba-2}

Este segundo caso de prueba se centra en una persona que ha superado todas 
las asignaturas del primer curso siguiendo el plan de estudios propuesto por 
la FIB. Se trata también de una persona con dedicación completa a sus estudios
universitarios y que, por lo tanto, no impone restricciones sobre la 
recomendación ni tiene ninguna preferencia particular.

Por lo tanto, parece que lo más razonable sería que el sistema recomendase 
que el alumno se matriculara de todas las asignaturas del tercer cuatrimestre 
según el plan de estudios (ya que el alumno parece tener capacidad y 
disponibilidad suficiente para asumir esta carga de trabajo y aún no se 
encuentra en disposición de elegir asignaturas de especialidad u optativas).

El objetivo de este segundo juego de pruebas es, pues, determinar si el 
sistema es capaz de ofrecer la recomendación ``obvia'' en un caso simple, 
teniendo en cuenta su conocimiento sobre la estructura del plan de estudios 
propuesto por la FIB.

\lstinputlisting[caption={Salida original del segundo juego de pruebas.},%
        label={lst:prueba-2}]%
    {resultado-2.txt}

El \autoref{lst:prueba-2} muestra la salida original del programa con este 
caso particular.
%%% TODO : Análisis del resultado obtenido y comparativa con lo esperado.


