
\subsection{Tercer juego de pruebas} \label{sec:prueba-3}

El tercer juego de pruebas es esencialmente la continuación natural del 
anterior. En este caso, se considera una persona que ha seguido bastante 
el plan de estudios propuesto por la FIB, aprobando todas las asignaturas en 
su primera convocatoria, gracias a su dedicación exclusiva a la universidad. 
Esta persona ha completado todas las asignaturas obligatorias excepto una 
y ya ha cursado tres asignaturas de la especialidad de computación. Como en 
los casos anteriores, esta persona no ofrece restricciones ni preferencias de 
forma explícita y deja esta tarea para el sistema de recomendación.

Teniendo en cuenta las asignaturas cursadas previamente, se espera que el 
sistema ofrezca una recomendación compuesta mayoritariamente por asignaturas 
de la especialidad de computación (por la que el alumno en cuestión ya ha 
demostrado interés). También sería lógico recomendar la asignatura obligatoria 
que le queda o, incluso, alguna optativa que guarde cierta relación con las 
asignaturas de la especialidad cursadas (si el sistema es capaz de inferir 
algún tema de particular interés a partir de estas).

Con esta entrada, pues, se pretende comprobar el tratamiento que el sistema 
construido es capaz de dar a los distintos tipos de asignatura (en particular, 
a las asignaturas de una especialidad concreta y a las optativas). Además, 
el hecho de incluir asignaturas de especialidad en este caso añade otro factor 
a tener en cuenta por el sistema: los temas de interés del alumno.

\lstinputlisting[caption={Salida original del tercer juego de pruebas.},%
        label={lst:prueba-3}]%
    {resultado-3.txt}

El \autoref{lst:prueba-3} muestra la salida del programa obtenida con este 
juego de pruebas.
%%% TODO : Análisis del resultado obtenido y comparativa con lo esperado.


