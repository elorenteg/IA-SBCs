
\subsection{Tercer juego de pruebas} \label{sec:prueba-3}

El tercer juego de pruebas es esencialmente la continuación natural del 
anterior. En este caso, se considera una persona que ha seguido bastante 
fielmente el plan de estudios propuesto por la FIB, aprobando todas las 
asignaturas en su primera convocatoria gracias a su dedicación exclusiva a la 
universidad. Esta persona ha completado todas las asignaturas obligatorias 
excepto una y ya ha cursado tres asignaturas de la especialidad de 
computación. Como en los casos anteriores, esta persona no ofrece 
restricciones ni preferencias de forma explícita y deja esta tarea para el 
sistema de recomendación.

Teniendo en cuenta las asignaturas cursadas previamente, se espera que el 
sistema ofrezca una recomendación compuesta mayoritariamente por asignaturas 
de la especialidad de computación (por la que el alumno en cuestión ya ha 
demostrado interés). También sería lógico recomendar la asignatura obligatoria 
que le queda o, incluso, alguna optativa que guarde cierta relación con las 
asignaturas de la especialidad cursadas (si el sistema es capaz de inferir 
algún tema de particular interés a partir de estas).

Con esta entrada, pues, se pretende comprobar el tratamiento que el sistema 
construido es capaz de dar a los distintos tipos de asignatura (en particular, 
a las asignaturas de una especialidad concreta y a las optativas). Además, 
el hecho de incluir asignaturas de especialidad en este caso añade otro factor 
a tener en cuenta por el sistema: los temas de interés del alumno.

\lstinputlisting[caption={Salida original del tercer juego de pruebas.},%
        label={lst:prueba-3}]%
    {resultado-3.txt}

El \autoref{lst:prueba-3} muestra la salida del programa obtenida con este 
juego de pruebas. En este caso, el sistema recomienda tres asignaturas de la 
especialidad de computación: dos de ellas son obligatorias y la otra es 
complementaria (se trata de la asignatura de compiladores; parece razonable 
que el sistema la considere adecuada para el alumno porque ha cursado la 
asignatura de lenguajes de programación previamente y ambas asignaturas están 
estrechamente relacionadas). Esta parte de la recomendación, pues, es muy 
parecida a lo que esperábamos al diseñar este juego de pruebas. Sin embargo, 
puede resultar sorprendente que la última asignatura de la recomendación sea 
una asignatura de la especialidad de tecnologías de la información (que se 
recomienda aquí en calidad de optativa). Esto se debe a que, por la falta de 
información proporcionada por el alumno, el sistema ha inferido temas de 
interés a partir de las asignaturas cursadas, la gran mayoría de las cuales 
son obligatorias. Y, casualmente, ha resultado que los temas tratados en la 
asignatura de internet móvil son parecidos a los temas tratados en algunas 
asignaturas cursadas por el alumno, por lo cual el sistema ha juzgado que 
esa asignatura podría interesarle. Quizás hubiera sido más acertado recomendar
la asignatura de paralelismo (la única asignatura obligatoria que le queda) 
u otra asignatura de la especialidad de computación, en vista de la falta de 
información acerca de los gustos del alumno.

En conclusión, la recomendación obtenida para este juego de pruebas es 
bastante buena en general (se recomiendan bastantes asignaturas de la 
especialidad cursada por el alumno e incluso se razona de forma acertada 
sobre las preferencias del alumno mostradas con las pocas asignaturas de 
especialidad completadas) pero se desvía ligeramente de lo esperado porque se 
ha forzado al sistema a inferir la mayor parte del conocimiento (y el proceso 
de inferencia puede dar lugar a equívocos en casos puntuales porque no hay 
forma perfecta de sacar conclusiones siempre correctas). 


