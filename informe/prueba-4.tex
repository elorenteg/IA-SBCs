
\subsection{Cuarto juego de pruebas} \label{sec:prueba-4}

En este cuarto juego de pruebas, se plantea el caso de un alumno que se 
encuentra cursando la especialidad de ingeniería del software. Sin embargo, 
esta persona está involucrada en actividades ajenas a la universidad que solo 
le permiten acudir a clase en horario de mañana (a pesar de que las 
asignaturas de su especialidad solo están disponibles por la tarde durante el 
cuatrimestre). Aun así, este sujeto dispone de una gran capacidad de trabajo 
y, por lo tanto, puede asumir la carga de trabajo correspondiente a una 
dedicación completa a la universidad.

Sin ser consciente de los horarios, la persona en cuestión utiliza el sistema 
para obtener una recomendación de asignaturas expresando explícitamente su 
interés en matricularse de asignaturas de la especialidad de ingeniería del 
software y sobre temas con una cierta relación con esta especialidad que le 
han interesado. Asimismo, muestra preferencia por algunas competencias 
transversales.

En estas circunstancias, se espera que el sistema sea capaz de recomendar 
asignaturas optativas (o bien asignaturas de otras especialidades que puedan 
satisfacer de algún modo las preferencias del alumno) relacionadas con los 
temas que interesan y que desarrollen las competencias transversales deseadas 
por esta persona. El sistema de inferencia debería determinar que se puede 
recomendar un gran número de asignaturas sin preocuparse demasiado por la 
carga de trabajo.

Con este juego de pruebas, pues, se puede comprobar el correcto 
comportamiento del sistema cuando se enfrenta a una mezcla de preferencias y 
restricciones explícitas además del proceso de inferencia.

\lstinputlisting[caption={Salida original del cuarto juego de pruebas.},%
        label={lst:prueba-4a}]%
    {resultado-4a.txt}

El \autoref{lst:prueba-4a} muestra los resultados obtenidos. Como se esperaba, 
el sistema ha recomendado principalmente asignaturas de otras especialidades 
(tecnologías de la información e ingeniería de computadores) en calidad de 
optativas para suplir la imposibilidad de cursar asignaturas de la 
especialidad de ingeniería del software en el horario deseado. Estas 
asignaturas, además, tratan los temas de interés que el usuario ha introducido 
como preferencias o bien desarrollan alguna de las competencias transversales 
deseadas. Además, se ha recomendado la única asignatura obligatoria que el 
alumno todavía no había cursado. La recomendación obtenida, pues, parece muy 
razonable teniendo en cuenta las fuertes restricciones con las que se ha 
encontrado el sistema en este caso. A pesar de todo, el sistema ha marcado 
estas asignaturas simplemente como recomendables (y no como altamente 
recomendables) porque encuentra relativamente pocos motivos para recomendarlas 
(aunque no haya otras asignaturas candidatas más acordes a los criterios 
evaluados). Esto alerta de que podría haber otras recomendaciones igualmente 
válidas en este caso.

Con este juego de pruebas, podemos concluir que el sistema es capaz de 
adaptarse y ofrecer un buen comportamiento incluso ante casos complejos en los 
que no hay una opción claramente superior.

\lstinputlisting[caption={Salida original de la versión alternativa del %
        cuarto juego de pruebas.}, label={lst:prueba-4b}]%
    {resultado-4b.txt}

Se ha considerado también una versión alternativa de este juego de pruebas 
para poner a prueba la robustez del sistema. En esta otra versión, el alumno 
impone como restricciones que tiene que completar la especialidad de 
ingeniería del software y que las asignaturas recomendadas deben estar 
disponibles en horario de tarde. Por lo tanto, no existe ninguna recomendación 
posible que se adecúe a los criterios requeridos y, por lo tanto, el sistema 
informa de esta situación, tal y como muestra el \autoref{lst:prueba-4b}.


