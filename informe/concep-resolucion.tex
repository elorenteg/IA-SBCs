
\subsection{Proceso de resolución} \label{sec:desc-resolucion}

Como se ha explicado en la \autoref{sec:subproblemas}, el proceso de 
resolución del problema empieza con una recopilación de información sobre las 
preferencias del alumno y las restricciones que este impone en las asignaturas 
que se le recomendarán. A tal efecto, se formulan una serie de preguntas para 
determinar la carga máxima de trabajo y la dificultad de las asignaturas que 
el alumno está dispuesto a asumir, sus preferencias o necesidades horarias, 
si quiere empezar una cierta especialidad o, en caso de que ya la haya 
empezado, si tiene especial interés en completarla, si da especial importancia 
a las competencias transversales que aún no ha obtenido, cuáles temas le 
interesan y cuáles no para las próximas asignaturas, ... El usuario puede 
decidir no responder a algunas de estas preguntas, en cuyo caso se adaptarán 
las preguntas restantes para evitar preguntar sobre temas similares a los 
ignorados.

Para todas aquellas preguntas que el usuario decida ignorar, se aplicará un 
proceso de inferencia basado en su expediente para obtener un conocimiento 
equivalente en la medida de lo posible. En particular, se observarán los 
patrones en la carga de trabajo y la dificultad de las asignaturas cursadas 
en cuatrimestres anteriores y los resultados académicos obtenidos para estimar 
las capacidades y la dedicación del alumno, se buscarán algunos patrones 
básicos en los horarios de estas asignaturas previas para deducir la 
disponibilidad del alumno, se evaluará su progreso en el plan de estudios del 
grado para determinar si le conviene cursar unas determinadas asignaturas 
(por ejemplo, para terminar la especialidad o bien para completar todas las 
competencias transversales y poder obtener el título del grado), se 
considerarán los temas tratados en las asignaturas que haya elegido 
previamente para determinar sus intereses, ...

Tras la adquisición de todo el conocimiento necesario para resolver el 
problema original, se procede a una abstracción de aquellas características 
más relevantes del alumno para simplificar la evaluación de las asignaturas 
recomendables. Es decir, planteamos la resolución de este problema usando las 
técnicas de clasificación heurística (sin embargo, con una fase de 
refinamiento prácticamente insignificante en comparación con el resto del 
problema porque la asociación heurística ya proporciona la solución).

Posteriormente, las asignaturas se filtran en función de sus características 
(horarios, dificultad, carga de trabajo, temas tratados, ...) con el objetivo 
de retirar aquellas que no cumplen alguna de las restricciones impuestas por 
el alumno. Finalmente, se valoran las asignaturas restantes mediante reglas 
heurísticas de razonamiento y se les asigna una puntuación que estima cuán 
recomendables son, incentivando la elección de aquellas que más se acercan 
a las preferencias del usuario. Durante este proceso, se pueden almacenar 
también los motivos que llevan al sistema a establecer las puntuaciones (es 
decir, las reglas que se han podido aplicar en función de las preferencias 
consideradas). Con esto, el problema ya está prácticamente resuelto y solo 
queda mostrar al usuario aquellas asignaturas con una puntuación más elevada 
(es decir, las más recomendables) junto a una calificación cualitativa del 
grado de recomendación y los motivos por los que se recomiendan. 
Concretamente, se eligen hasta un máximo de seis asignaturas y todas ellas 
deben ser mínimamente apropiadas para el perfil del alumno.



