
\subsection{Quinto juego de pruebas} \label{sec:prueba-5}

El quinto sujeto de pruebas es alguien que trabaja a la vez que estudia. Por 
ello, se trata de una persona con una baja disponibilidad temporal para la 
universidad y con dificultades para asumir grandes cargas de trabajo. Esta 
persona ha iniciado ya la especialidad de computación, pero solo ha 
completado un par de asignaturas de esta porque en los últimos cuatrimestres 
ha podido matricularse de muy pocas asignaturas. Por otro lado, el alumno 
tiene mucho interés en algunos temas relacionados con la especialidad de 
computación y, además, le gustaría desarrollar algunas competencias 
transversales en concreto.

Consideramos dos versiones de este juego de pruebas. En la primera de ellas, 
el alumno impone restricciones para garantizar que la recomendación que 
obtiene supone una carga de trabajo reducida y que se tratan los temas que 
más le apasionan. En la segunda, en cambio, el alumno confía en el sistema de 
inferencia para detectar su baja disponibilidad (teniendo en cuenta la carga 
de trabajo asumida en los cuatrimestres previos) y simplemente expresa como 
preferencias los temas y las competencias transversales que le interesan, 
dando así una mayor libertad al sistema de recomendación. En ambas versiones,
las asignaturas recomendadas tienen que ofrecerse en horario de tarde, ya que 
esta persona trabaja en una empresa por la mañana.

Este juego de pruebas pretende contrastar las recomendaciones obtenidas cuando 
se imponen restricciones estrictas respecto a cuando se deja que el sistema 
infiera preferencias basadas en el conocimiento pasado. En el primero de los 
casos, se espera que la recomendación se adecúe totalmente a las 
necesidades del alumno aunque sea bastante limitada; en el segundo caso, en 
contraposición, se espera obtener una recomendación más amplia pero menos 
precisa.

\lstinputlisting[caption={Salida original de la versión con restricciones del %
        quinto juego de pruebas.}, label={lst:prueba-5a}]%
    {resultado-5a.txt}

El \autoref{lst:prueba-5a} muestra la salida obtenida con la versión más 
restrictiva de este juego de pruebas. En esta versión, el usuario ha impuesto 
restricciones explícitas para señalar su baja disponibilidad y que desea que 
las asignaturas que matricule traten sobre lógica e inteligencia artificial. 
Además de esto, expresa su preferencia por matricularse de un máximo de tres 
asignaturas y por desarrollar las competencias transversales de trabajo en 
equipo y de razonamiento. Por lo tanto, la recomendación ofrecida por el 
sistema es muy acertada: las dos asignaturas recomendadas son de la 
especialidad cursada por el alumno y, entre las dos, tratan los dos temas y 
desarrollan las dos competencias transversales por los que ha mostrado 
interés. Además, se respetan la disponibilidad horaria y la imposibilidad de 
asumir cargas de trabajo intensas. 

En este caso, pues, el sistema responde a la perfección a las necesidades del 
alumno, aunque es muy conservador en la recomendación debido a las 
restricciones impuestas.

\lstinputlisting[caption={Salida original de la versión con preferencias e % 
        inferencia del quinto juego de pruebas.}, label={lst:prueba-5b}]%
    {resultado-5b.txt}

El \autoref{lst:prueba-5b} muestra la salida obtenida con la versión menos 
restrictiva de este juego de pruebas. A diferencia de la versión anterior, en 
esta el usuario expresa sus intereses temáticos y las limitaciones debidas a 
su trabajo como preferencias (y no como restricciones). Por este motivo, la 
recomendación ofrecida en este caso es menos conservadora y, tal vez, menos 
precisa también. En este caso, se recomiendan tres asignaturas de la 
especialidad de computación bastante relacionadas con los temas y competencias 
transversales de interés, pero no se incluye ninguna asignatura de lógica 
(porque no es un requisito, sino una preferencia simplemente). 

En conclusión, observamos que el sistema ofrece recomendaciones más precisas 
pero conservadoras (en el sentido que la cantidad de asignaturas que se tienen 
en cuenta para la recomendación es bastante más limitada) cuando se imponen 
más restricciones. En cambio, cuando estas limitaciones se expresan como 
preferencias, el sistema es capaz de tener en cuenta más asignaturas para la 
recomendación y, por ello, también puede ofrecer una recomendación que cumpla 
menos criterios de los deseados.


