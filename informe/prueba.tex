
\section{Prueba} \label{sec:prueba}

Finalmente, tras la implementación del sistema basado en el conocimiento que 
resuelve el problema de recomendación de asignaturas, solamente queda 
comprobar y estudiar las soluciones obtenidas para instancias concretas del 
problema. Así, a continuación se muestran y analizan los resultados logrados 
mediante algunos juegos de prueba representativos.

Los juegos de prueba seleccionados pretenden emular las distintas situaciones 
en las que se pueden encontrar los alumnos de la FIB. Concretamente, se han 
planteado tres casos de alumnos que se encuentran en distintas fases del plan 
de estudios y que prácticamente no tienen restricciones ni preferencias muy 
estrictas (estos casos deberían resultar relativamente sencillos de resolver 
para el programa) y tres casos más complicados de alumnos con algunas 
circunstancias más excepcionales (estos casos deberían ser más difíciles de 
resolver, puesto que hay más factores a tener en cuenta, y deberían servir 
para evaluar la capacidad del sistema desarrollado para ofrecer 
recomendaciones adecuadas).


\subsection{Primer juego de pruebas} \label{sec:prueba-1}

%%% TODO : Explicación del juego de pruebas y análisis de lo que se espera.

\lstinputlisting[caption={Salida original del primer juego de pruebas.},%
        label={lst:prueba-1}]%
    {resultado-1.txt}

%%% TODO : Análisis del resultado obtenido y comparativa con lo esperado.





\subsection{Segundo juego de pruebas} \label{sec:prueba-2}

Este segundo caso de prueba se centra en una persona que ha superado todas 
las asignaturas del primer curso siguiendo el plan de estudios propuesto por 
la FIB. Se trata también de una persona con dedicación completa a sus estudios
universitarios y que, por lo tanto, no impone restricciones sobre la 
recomendación ni tiene ninguna preferencia particular.

Por lo tanto, parece que lo más razonable sería que el sistema recomendase 
que el alumno se matriculara de todas las asignaturas del tercer cuatrimestre 
según el plan de estudios (ya que el alumno parece tener capacidad y 
disponibilidad suficiente para asumir esta carga de trabajo y aún no se 
encuentra en disposición de elegir asignaturas de especialidad u optativas).

El objetivo de este segundo juego de pruebas es, pues, determinar si el 
sistema es capaz de ofrecer la recomendación ``obvia'' en un caso simple, 
teniendo en cuenta su conocimiento sobre la estructura del plan de estudios 
propuesto por la FIB.

\lstinputlisting[caption={Salida original del segundo juego de pruebas.},%
        label={lst:prueba-2}]%
    {resultado-2.txt}

El \autoref{lst:prueba-2} muestra la salida original del programa con este 
caso particular.
%%% TODO : Análisis del resultado obtenido y comparativa con lo esperado.





\subsection{Tercer juego de pruebas} \label{sec:prueba-3}

El tercer juego de pruebas es esencialmente la continuación natural del 
anterior. En este caso, se considera una persona que ha seguido bastante 
fielmente el plan de estudios propuesto por la FIB, aprobando todas las 
asignaturas en su primera convocatoria gracias a su dedicación exclusiva a la 
universidad. Esta persona ha completado todas las asignaturas obligatorias 
excepto una y ya ha cursado tres asignaturas de la especialidad de 
computación. Como en los casos anteriores, esta persona no ofrece 
restricciones ni preferencias de forma explícita y deja esta tarea para el 
sistema de recomendación.

Teniendo en cuenta las asignaturas cursadas previamente, se espera que el 
sistema ofrezca una recomendación compuesta mayoritariamente por asignaturas 
de la especialidad de computación (por la que el alumno en cuestión ya ha 
demostrado interés). También sería lógico recomendar la asignatura obligatoria 
que le queda o, incluso, alguna optativa que guarde cierta relación con las 
asignaturas de la especialidad cursadas (si el sistema es capaz de inferir 
algún tema de particular interés a partir de estas).

Con esta entrada, pues, se pretende comprobar el tratamiento que el sistema 
construido es capaz de dar a los distintos tipos de asignatura (en particular, 
a las asignaturas de una especialidad concreta y a las optativas). Además, 
el hecho de incluir asignaturas de especialidad en este caso añade otro factor 
a tener en cuenta por el sistema: los temas de interés del alumno.

\lstinputlisting[caption={Salida original del tercer juego de pruebas.},%
        label={lst:prueba-3}]%
    {resultado-3.txt}

El \autoref{lst:prueba-3} muestra la salida del programa obtenida con este 
juego de pruebas. En este caso, el sistema recomienda tres asignaturas de la 
especialidad de computación: dos de ellas son obligatorias y la otra es 
complementaria (se trata de la asignatura de compiladores; parece razonable 
que el sistema la considere adecuada para el alumno porque ha cursado la 
asignatura de lenguajes de programación previamente y ambas asignaturas están 
estrechamente relacionadas). Esta parte de la recomendación, pues, es muy 
parecida a lo que esperábamos al diseñar este juego de pruebas. Sin embargo, 
puede resultar sorprendente que la última asignatura de la recomendación sea 
una asignatura de la especialidad de tecnologías de la información (que se 
recomienda aquí en calidad de optativa). Esto se debe a que, por la falta de 
información proporcionada por el alumno, el sistema ha inferido temas de 
interés a partir de las asignaturas cursadas, la gran mayoría de las cuales 
son obligatorias. Y, casualmente, ha resultado que los temas tratados en la 
asignatura de internet móvil son parecidos a los temas tratados en algunas 
asignaturas cursadas por el alumno, por lo cual el sistema ha juzgado que 
esa asignatura podría interesarle. Quizás hubiera sido más acertado recomendar
la asignatura de paralelismo (la única asignatura obligatoria que le queda) 
u otra asignatura de la especialidad de computación, en vista de la falta de 
información acerca de los gustos del alumno.

En conclusión, la recomendación obtenida para este juego de pruebas es 
bastante buena en general (se recomiendan bastantes asignaturas de la 
especialidad cursada por el alumno e incluso se razona de forma acertada 
sobre las preferencias del alumno mostradas con las pocas asignaturas de 
especialidad completadas) pero se desvía ligeramente de lo esperado porque se 
ha forzado al sistema a inferir la mayor parte del conocimiento (y el proceso 
de inferencia puede dar lugar a equívocos en casos puntuales porque no hay 
forma perfecta de sacar conclusiones siempre correctas). 





\subsection{Cuarto juego de pruebas} \label{sec:prueba-4}

%%% TODO : Explicación del juego de pruebas y análisis de lo que se espera.

\lstinputlisting[caption={Salida original del cuarto juego de pruebas.},%
        label={lst:prueba-4}]%
    {resultado-4.txt}

%%% TODO : Análisis del resultado obtenido y comparativa con lo esperado.





\subsection{Quinto juego de pruebas} \label{sec:prueba-5}

El quinto sujeto de pruebas es alguien que trabaja a la vez que estudia. Por 
ello, se trata de una persona con una baja disponibilidad temporal para la 
universidad y con dificultades para asumir grandes cargas de trabajo. Esta 
persona ha iniciado ya la especialidad de computación, pero solo ha 
completado un par de asignaturas de esta porque en los últimos cuatrimestres 
ha podido matricularse de muy pocas asignaturas. Por otro lado, el alumno 
tiene mucho interés en algunos temas relacionados con la especialidad de 
computación y, además, le gustaría desarrollar algunas competencias 
transversales en concreto.

Consideramos dos versiones de este juego de pruebas. En la primera de ellas, 
el alumno impone restricciones para garantizar que la recomendación que 
obtiene supone una carga de trabajo reducida y que se tratan los temas que 
más le apasionan. En la segunda, en cambio, el alumno confía en el sistema de 
inferencia para detectar su baja disponibilidad (teniendo en cuenta la carga 
de trabajo asumida en los cuatrimestres previos) y simplemente expresa como 
preferencias los temas y las competencias transversales que le interesan, 
dando así una mayor libertad al sistema de recomendación. En ambas versiones,
las asignaturas recomendadas tienen que ofrecerse en horario de tarde, ya que 
esta persona trabaja en una empresa por la mañana.

Este juego de pruebas pretende contrastar las recomendaciones obtenidas cuando 
se imponen restricciones estrictas respecto a cuando se deja que el sistema 
infiera preferencias basadas en el conocimiento pasado. En el primero de los 
casos, se espera que la recomendación se adecúe totalmente a las 
necesidades del alumno aunque sea bastante limitada; en el segundo caso, en 
contraposición, se espera obtener una recomendación más amplia pero menos 
precisa.

\lstinputlisting[caption={Salida original de la versión con restricciones del %
        quinto juego de pruebas.}, label={lst:prueba-5a}]%
    {resultado-5a.txt}

El \autoref{lst:prueba-5a} muestra la salida obtenida con la versión más 
restrictiva de este juego de pruebas. En esta versión, el usuario ha impuesto 
restricciones explícitas para señalar su baja disponibilidad y que desea que 
las asignaturas que matricule traten sobre lógica e inteligencia artificial. 
Además de esto, expresa su preferencia por matricularse de un máximo de tres 
asignaturas y por desarrollar las competencias transversales de trabajo en 
equipo y de razonamiento. Por lo tanto, la recomendación ofrecida por el 
sistema es muy acertada: las dos asignaturas recomendadas son de la 
especialidad cursada por el alumno y, entre las dos, tratan los dos temas y 
desarrollan las dos competencias transversales por los que ha mostrado 
interés. Además, se respetan la disponibilidad horaria y la imposibilidad de 
asumir cargas de trabajo intensas. 

En este caso, pues, el sistema responde a la perfección a las necesidades del 
alumno, aunque es muy conservador en la recomendación debido a las 
restricciones impuestas.

\lstinputlisting[caption={Salida original de la versión con preferencias e % 
        inferencia del quinto juego de pruebas.}, label={lst:prueba-5b}]%
    {resultado-5b.txt}

El \autoref{lst:prueba-5b} muestra la salida obtenida con la versión menos 
restrictiva de este juego de pruebas. A diferencia de la versión anterior, en 
esta el usuario expresa sus intereses temáticos y las limitaciones debidas a 
su trabajo como preferencias (y no como restricciones). Por este motivo, la 
recomendación ofrecida en este caso es menos conservadora y, tal vez, menos 
precisa también. En este caso, se recomiendan tres asignaturas de la 
especialidad de computación bastante relacionadas con los temas y competencias 
transversales de interés, pero no se incluye ninguna asignatura de lógica 
(porque no es un requisito, sino una preferencia simplemente). 

En conclusión, observamos que el sistema ofrece recomendaciones más precisas 
pero conservadoras (en el sentido que la cantidad de asignaturas que se tienen 
en cuenta para la recomendación es bastante más limitada) cuando se imponen 
más restricciones. En cambio, cuando estas limitaciones se expresan como 
preferencias, el sistema es capaz de tener en cuenta más asignaturas para la 
recomendación y, por ello, también puede ofrecer una recomendación que cumpla 
menos criterios de los deseados.





\subsection{Sexto juego de pruebas} \label{sec:prueba-6}

Este último sujeto de prueba es una persona en su último curso del grado. 
Esta persona ha completado todas las asignaturas obligatorias y todas las de 
la especialidad de computación siguiendo el plan de estudios propuesto por 
la FIB y, además, obteniendo calificaciones bastante buenas. Por ello, este 
alumno desea que se le recomienden asignaturas optativas acordes a sus gustos 
y que desarrollen las competencias transversales que todavía no ha obtenido. 
Además, este alumno ya empieza a pensar en su futuro fuera de la universidad 
y, por lo tanto, tiene algunas preocupaciones adicionales. 

Consideramos también dos versiones de este juego de pruebas. En la primera de 
ellas, el alumno está especialmente preocupado por su insuficiente 
conocimiento del inglés y el impacto que esto podría tener en su futuro 
laboral. Por este motivo, impone como restricción estricta que las asignaturas 
de las que quiere matricularse desarrollen la competencia transversal de la 
tercera lengua. Además, prefiere no tener una carga de trabajo excesiva para 
poder ir pensando en el trabajo de final de grado, así que impone algunas 
restricciones (no muy restrictivas) en cuanto a la dedicación a la 
universidad. Además de ello, el alumno muestra un cierto interés por algunos 
temas concretos. Por contra, en la segunda versión, el alumno se plantea la 
posibilidad de desarrollar el trabajo de final de grado en inglés y, por lo 
tanto, ya no es tan importante que las asignaturas trabajen la tercera lengua. 
En este caso, pues, la restricción sobre la competencia transversal pasa a 
ser una preferencia y ya no se imponen restricciones sobre la carga de trabajo
asumible.

Teniendo en cuenta la situación del alumno, se espera que se recomienden 
principalmente asignaturas optativas (o de alguna especialidad que se puedan 
cursar en concepto de optativas) sobre algunos de los temas que interesan al 
alumno y que desarrollen algunas de las competencias transversales preferidas, 
dando una especial importancia a la competencia de la tercera lengua en la 
primera versión.

Con este juego de pruebas, pues, se intenta confirmar el distinto 
comportamiento del sistema frente a entradas más o menos restrictivas, 
reforzando así los resultados observados en el juego de pruebas anterior. 
Además, este caso es sustancialmente distinto a los anteriores porque este 
alumno se encuentra ya cerca del final del plan de estudios y, por lo tanto, 
solo se pueden tener en cuenta asignaturas optativas, de otras especialidades 
o incluso alguna de la especialidad de computación que no se haya cursado. Es 
decir, el conjunto de asignaturas a tener en cuenta es mucho más reducido (a 
pesar de que no hay restricciones impuestas por el plan de estudios sobre 
requisitos o la necesidad de completar una especialidad).

\lstinputlisting[caption={Salida original de la primera versión del sexto %
        juego de pruebas.}, label={lst:prueba-6a}]%
    {resultado-6a.txt}

Los resultados obtenidos con la primera versión de este juego de pruebas se 
muestran en \autoref{lst:prueba-6a}. Se observa que, en este caso, el sistema 
ha dado mucha prioridad a las asignaturas que desarrollan la tercera lengua 
y ha sugerido una recomendación muy conservadora en este sentido. Así, las 
cuatro asignaturas recomendadas son optativas y tres de ellas están enfocadas 
básicamente a habilidades importantes para el mundo laboral y cursadas en 
inglés. Sin embargo, esto ha hecho que prácticamente se ignoren los temas de 
interés del alumno (solamente la asignatura de geometría computacional tiene 
en cuenta los intereses temáticos) y el resto de competencias transversales.
En definitiva, se trata de una recomendación más limitada de lo esperado en 
un principio, pero se puede entender por la importancia que da el sistema a 
las restricciones.

\lstinputlisting[caption={Salida original de la segunda versión del sexto % 
        juego de pruebas.}, label={lst:prueba-6b}]%
    {resultado-6b.txt}

El \autoref{lst:prueba-6b} muestra los resultados obtenidos con la versión 
menos restrictiva. En este caso, como era de esperar, se da una mayor 
importancia a los intereses temáticos del alumno y, en consecuencia, se 
recomiendan muchas asignaturas de la especialidad de computación basadas en 
los temas listados en las preferencias. En este caso, sin embargo, las 
competencias transversales preferidas han tenido un peso prácticamente nulo 
en la recomendación porque la mayoría de ellas ya se habían obtenido con 
las asignaturas cursadas previamente (y el sistema ha sido capaz de detectarlo 
a partir del expediente). Además, hay un cierto solapamiento entre las 
asignaturas recomendadas (tres de ellas se recomiendan por tratar el tema de 
tratamiento de datos). Esto se debe a las limitaciones del sistema a la hora 
de considerar los subconjuntos de asignaturas válidos para la recomendación 
(puesto que hay un gran número de ellos y, por lo tanto, el sistema aplica 
razonamientos heurísticos y voraces para obtener recomendaciones válidas en 
un tiempo factible a la práctica) y a su mayor facilidad para evaluar las 
asignaturas de forma individual. A pesar de todo, la recomendación presentada 
sigue siendo aceptable.

Con este juego de pruebas más complicado, hemos observado algunas limitaciones 
del sistema de recomendación. Así, en casos más complejos o con mayores 
restricciones, el sistema recomienda muchas asignaturas parecidas entre ellas 
y puede obviar algunas de las preferencias (porque se prioriza el número de 
preferencias satisfechas por asignatura). Aun así, es capaz de ofrecer 
recomendaciones aceptables, aunque no sean las mejores posibles desde el punto 
de vista de un experto.






%%% TODO %%%

%- Per al primer prototip, pensem en tres persones que segueixen el pla
%d'estudis i sense restriccions horàries ni res per l'estil. En
%particular, havia pensat en una persona que ha fet el primer
%quadrimestre i ha suspès la meitat de les assignatures (esperaria que la
%recomanació fos tornar a matricular les assignatures suspeses i fer-ne
%una o dues de les del segon quadrimestre), una persona que ha acabat el
%primer curs amb tot aprovat (esperaria que se li recomanessin totes les
%assignatures pròpies del tercer quadrimestre), i una persona que està a
%la meitat d'una especialitat i tampoc té problemes amb la dificultat o
%la càrrega de treball (esperaria que la recomanació fossin les
%assignatures d'especialitat que li queden).
%- Com a casos una mica més "complicats", havia pensat en una persona
%que ja està a mitja especialitat però no li van bé els horaris d'aquella
%especialitat (esperaria que se li recomanessin optatives de temes
%"semblants" a ser possible), una persona a qui li queden encara moltes
%assignatures d'especialitat però que en quadrimestres anteriors se n'ha
%hagut d'anar matriculant poques (esperaria que se li recomanessin poques
%assignatures, segons la càrrega de treball que pot suportar) i una
%persona que ja ha acabat l'especialitat i només li queden optatives i li
%falta alguna competència transversal (esperaria que se li recomanessin
%assignatures amb les competències transversals que li queden i de temes
%"semblants" als que ha estudiat).


\clearpage

