
\section{Implementación} \label{sec:implementacion}

En la \autoref{sec:formalizacion}, se ha descrito de forma detallada la 
versión final del sistema basado en el conocimiento que resuelve este problema 
y que se encuentra implementado en el código adjunto a este documento. En esta 
sección, pues, se describen las distintas fases del proceso de implementación.

Para implementar el sistema, se ha optado por una metodología de prototipado 
rápido incremental. Esta metodología consiste en la realización de prototipos 
funcionales en los que se van añadiendo funcionalidades del sistema final. Es 
decir, partiendo de un sistema básico de funcionalidad muy limitada, se han 
ido añadiendo elementos a tener en cuenta en el proceso de resolución hasta 
llegar al sistema final ya descrito. Con este método, pues, hemos podido tener 
un sistema funcional desde una fase temprana del desarrollo y que se ha podido 
probar a medida que se expandía con nuevas características. Además, esto ha 
facilitado las mejoras basadas en los resultados obtenidos con distintos 
juegos de prueba. En alguna de estas fases de expansión, ha sido necesario 
hacer alguna modificación menor de la ontología, como ya se ha explicado en la 
\autoref{sec:ontologia}. Aun así, el esmero en obtener una especificación 
correcta de los elementos del dominio ha permitido desarrollar sin problemas 
el sistema basado en la ontología desarrollada esencialmente durante las 
primeras semanas.

Inicialmente, se desarrolló un primer prototipo que sirvió para evaluar el 
funcionamiento y las posibilidades del lenguaje \texttt{CLIPS}. Esta primera 
versión constaba de dos módulos simples de funcionalidad reducida. El primero
de ellos constaba de funciones para hacer preguntas al usuario sobre algunas 
de sus preferencias y restricciones y añadir las respuestas obtenidas al 
sistema en forma de hechos estructurados con \texttt{deftemplates}. El segundo 
módulo generaba una recomendación prácticamente aleatoria, sin tener en cuenta 
de ninguna manera las preferencias o restricciones del alumno, simplemente 
elegía hasta seis asignaturas que no se hubiesen cursado anteriormente. Esto 
sirvió para aprender a navegar en las reglas de \texttt{CLIPS} entre los 
diferentes conceptos representados en la ontología.

Un segundo prototipo añadió el tratamiento de algunas preferencias y 
restricciones en la generación de la recomendación. En particular, se añadió 
un módulo para la fase de abstracción y otro módulo para la fase de asociación
(fases explicadas en la \autoref{sec:razonamiento}). En esta versión, sin 
embargo, solo se trataban las restricciones y preferencias de horario y de la 
especialidad cursada (en caso de que se introdujesen explícitamente como 
respuestas a las preguntas del sistema, ya que no había inferencia a partir 
del expediente todavía), puesto que eran las más sencillas de evaluar. Por lo 
tanto, el último módulo del primer prototipo se modificó para que eligiese 
asignaturas que cumplieran estas restricciones y preferencias (dentro de lo 
posible) y lo mostrase explícitamente (es decir, que diese una breve 
descripción de los motivos de la elección). 

En el tercer prototipo, se expandió el conjunto de restricciones y 
preferencias que el sistema era capaz de tener en cuenta en la recomendación. 
Así, el módulo de preguntas se mejoró con la inclusión de algunas preguntas 
addicionales para poder obtener más información del alumno. Asimismo, se 
añadieron las reglas pertinentes a los módulos de abstracción y de asociación 
para poder determinar qué restricciones y preferencias cumplían las distintas 
asignaturas a recomendar. En esta fase, ya se trataban todas las restricciones 
y preferencias descritas en la \autoref{sec:razonamiento} exceptuando las del 
refinamiento final. Por ello, el último módulo también se adaptó para 
recomendar aquellas asignaturas que cumplían todas estas restricciones y el 
mayor número posible de preferencias, detallando los motivos de la elección. 

El cuarto prototipo supuso un gran avance hacia la versión final. En este 
prototipo, se añadieron un módulo de inferencia de preferencias a partir del 
expediente del alumno, así como un módulo de refinamiento de soluciones. El 
módulo de inferencia contiene las reglas que crean hechos para expresar 
preferencias del alumno que no ha introducido explícitamente y que se deducen 
de su expediente. Por otro lado, el módulo de refinamiento incluye el 
tratamiento de las restricciones y preferencias que conciernen a la 
recomendación final o que requieren de conocimiento concreto, como se explica 
en la \autoref{sec:razonamiento}. 

Finalmente, a partir de este último prototipo, se construyó la versión final 
del sistema. Esta versión final incluyó ligeras mejoras en las reglas y en su 
estructura. Asimismo, se mejoró el módulo de selección de las asignaturas a 
recomendar para tener en cuenta todas las restricciones y preferencias. 
También se modificó el sistema de preguntas al usuario para hacerlo más 
``amigable'' y se expresó la salida (es decir, la recomendación final de 
asignaturas) de forma más ordenada y detallada. La estructura en módulos del 
programa final se asemeja a la división en subproblemas explicada en 
\autoref{sec:subproblemas}.


\clearpage

