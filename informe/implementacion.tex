
\section{Implementación} \label{sec:implementacion}

En la \autoref{sec:formalizacion}, se ha descrito de forma detallada la 
versión final del sistema basado en el conocimiento que resuelve este problema 
y que se encuentra implementado en el código adjunto a este documento. En esta 
sección, pues, se describen las distintas fases del proceso de implementación.

Para implementar el sistema, se ha optado por una metodología de prototipado 
rápido incremental. Esta metodología consiste en la realización de prototipos 
funcionales en los que se van añadiendo funcionalidades del sistema final. Es 
decir, partiendo de un sistema básico de funcionalidad muy limitada, se han 
ido añadiendo elementos a tener en cuenta en el proceso de resolución hasta 
llegar al sistema final ya descrito. Con este método, pues, hemos podido tener 
un sistema funcional desde una fase temprana del desarrollo y que se ha podido 
probar a medida que se expandía con nuevas características. Además, esto ha 
facilitado las mejoras basadas en los resultados obtenidos con distintos 
juegos de prueba. En alguna de estas fases de expansión, ha sido necesario 
hacer alguna modificación menor de la ontología, como ya se ha explicado en la 
\autoref{sec:ontologia}. Aun así, el esmero en obtener una especificación 
correcta de los elementos del dominio ha permitido desarrollar sin problemas 
el sistema basado en la ontología desarrollada esencialmente durante las 
primeras semanas.

Inicialmente, se desarrolló un primer prototipo que sirvió para evaluar el 
funcionamiento y las posibilidades del lenguaje \texttt{CLIPS}. Esta primera 
versión constaba de dos módulos simples de funcionalidad reducida. El primero
de ellos hacía preguntas al usuario sobre algunas de sus preferencias y 
restricciones y añadía las respuestas obtenidas al sistema en forma de hechos 
estructurados con \texttt{deftemplates}. El segundo módulo generaba una 
recomendación prácticamente aleatoria, sin tener en cuenta de ninguna manera 
las preferencias o restricciones del alumno, simplemente elegía hasta seis 
asignaturas que no se hubiesen cursado anteriormente. Esto sirvió para 
aprender a navegar en las reglas de \texttt{CLIPS} entre los diferentes 
conceptos representados en la ontología.

% TODO : Más prototipos. Por ejemplo: añadir módulos de abstracción y 
%        asociación incompletos, añadir más reglas, añadir fase de refinamiento.


\clearpage

