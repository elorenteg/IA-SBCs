
\subsection{Razonamiento para la resolución} \label{sec:razonamiento}

Una vez analizados informalmente todos los elementos que constituyen el 
problema y establecida una representación formal de todo el conocimiento 
necesario sobre el dominio (es decir, la ontología), se puede justificar 
formalmente que el método de resolución señalado es adecuado y detallarlo 
aún más.

Como se ha indicado en la \autoref{sec:viabilidad}, el problema al que nos 
enfrentamos es un problema de análisis, en el que hay que escoger una solución
de entre un conjunto finito: en particular, hay que seleccionar el subconjunto 
de asignaturas más adecuado para el alumno en función de sus preferencias y 
restricciones. Para ello, pues, el sistema debe evaluar la adecuación de 
cada asignatura a un estudiante teniendo en cuenta el conocimiento del que 
dispone sobre las asignaturas, el plan de estudios y el alumno, y así 
clasificar las asignaturas en recomendables o no recomendables (es decir, hay 
que interpretar los datos de entrada del problema para poder seleccionar una 
solución adecuada; esta es precisamente una caracterización de los problemas 
de análisis).

Además, partiendo de unos datos de entrada que proporcionan un conocimiento 
incompleto hay que asociar a estos una solución mediante razonamientos 
heurísticos. En conclusión, el método más adecuado para resolver este problema 
es la clasificación heurística. Para nuestro problema de recomendación la 
resolución consistirá en tres fases:

\begin{enumerate}

\item La abstracción de datos consiste en pasar de un problema concreto, 
la situación de nuestro alumno, a un problema abstracto, una generalización 
para tratar a todos los alumnos. En esta fase, pues, incluimos los 
subproblemas de obtención de conocimiento mediante preguntas al usuario y la 
inferencia del conocimiento restante a partir de su expediente. 
Si el alumno nos ha introducido preferencias y restricciones, la abstracción 
será sencilla porque ya tendremos toda la información que necesitaremos para 
el sistema. En caso contrario, deberemos deducir sus preferencias según su 
expediente mediante un sistema de reglas heurísticas (aproximaciones basadas  
en nuestro conocimiento del dominio). Así, en esta fase de abstracción, 
simplificamos todo este conocimiento concreto en un conjunto mucho más 
reducido (también más genérico e impreciso) que permitirá un tratamiento 
más eficiente para la generación de soluciones. Esta última parte se 
corresponde con el subproblema de abstracción de características clave.

\item La asociación heurística consiste en pasar de un problema abstracto a 
una solución abstracta. En esta fase se quiere obtener una posible solución 
que normalmente ha funcionado para un mismo patrón de estudiantes y del que 
se genera una serie de recomendaciones de asignaturas según si son 
altamente recomendables o solo recomendables. Más específicamente, en esta 
fase se usan una serie de reglas de razonamiento deductivo. Estas reglas, a 
partir del conocimiento simplificado de la situación del alumno obtenido en 
la fase previa, califican la adecuación de cada asignatura al alumno de un 
modo que puede no ser totalmente preciso (puesto que se han omitido detalles 
para conseguir un método de resolución más eficiente). Por otro lado, se 
mantiene un registro con las reglas que se han podido aplicar para cada 
asignatura: este registro constituye un resumen de los motivos que llevan a 
la calificación final. En esta fase, pues, se incluye el subproblema de 
evaluación del grado de recomendación.

\item El refinamiento y la adaptación consisten en pasar de una solución 
abstracta a una solución concreta. De este modo, podremos pasar de una 
solución con distintas recomendaciones, que nos encajaría con el mismo 
tipo de estudiante, a nuestro alumno. En nuestro caso particular, escogeremos 
hasta un máximo de seis asignaturas entre todas las recomendadas por el 
sistema según los datos concretos del alumno para poder depurar mejor la 
recomendación. Concretamente, en esta fase se seleccionan las asignaturas que 
se mostrarán al usuario junto a los motivos por los cuales se recomiendan. 
Además, se garantiza (esta vez teniendo en cuenta todo el conocimiento 
concreto que se tiene del alumno) que la recomendación final cumple todas las 
restricciones requeridas. Esta fase, pues, corresponde al último de los 
subproblemas establecidos.

\end{enumerate}

%%% TODO : Explicación del "proceso de razonamiento de los subproblemas".
%%%        Entiendo que se trata de describir un poco a alto nivel las reglas
%%%        con las que resolvemos el problema.


