
\subsection{Razonamiento para la resolución} \label{sec:razonamiento}

Una vez analizados informalmente todos los elementos que constituyen el 
problema y establecida una representación formal de todo el conocimiento 
necesario sobre el dominio (es decir, la ontología), se puede justificar 
formalmente que el método de resolución señalado es adecuado y detallarlo 
aún más.

Como se ha indicado en la \autoref{sec:viabilidad}, el problema al que nos 
enfrentamos es un problema de análisis, en el que hay que escoger una solución
de entre un conjunto finito: en particular, hay que seleccionar el subconjunto 
de asignaturas más adecuado para el alumno en función de sus preferencias y 
restricciones. Para ello, pues, el sistema debe evaluar la adecuación de 
cada asignatura a un estudiante teniendo en cuenta el conocimiento del que 
dispone sobre las asignaturas, el plan de estudios y el alumno, y así 
clasificar las asignaturas en recomendables o no recomendables (es decir, hay 
que interpretar los datos de entrada del problema para poder seleccionar una 
solución adecuada; esta es precisamente una caracterización de los problemas 
de análisis).

Además, partiendo de unos datos de entrada que proporcionan un conocimiento 
incompleto hay que asociar a estos una solución mediante razonamientos 
heurísticos. En conclusión, el método más adecuado para resolver este problema 
es la clasificación heurística. Para nuestro problema de recomendación la 
resolución consistirá en tres fases:

\begin{enumerate}

\item La abstracción de datos consiste en pasar de un problema concreto, 
la situación de nuestro alumno, a un problema abstracto, una generalización 
para tratar a todos los alumnos. En esta fase, pues, incluimos los 
subproblemas de obtención de conocimiento mediante preguntas al usuario y la 
inferencia del conocimiento restante a partir de su expediente. 
Si el alumno nos ha introducido preferencias y restricciones, la abstracción 
será sencilla porque ya tendremos toda la información que necesitaremos para 
el sistema. En caso contrario, deberemos deducir sus preferencias según su 
expediente mediante un sistema de reglas heurísticas (aproximaciones basadas  
en nuestro conocimiento del dominio). Así, en esta fase de abstracción, 
simplificamos todo este conocimiento concreto en un conjunto mucho más 
reducido (también más genérico e impreciso) que permitirá un tratamiento 
más eficiente para la generación de soluciones. Esta última parte se 
corresponde con el subproblema de abstracción de características clave.

\item La asociación heurística consiste en pasar de un problema abstracto a 
una solución abstracta. En esta fase se quiere obtener una posible solución 
que normalmente ha funcionado para un mismo patrón de estudiantes y del que 
se genera una serie de recomendaciones de asignaturas según si son 
altamente recomendables o solo recomendables. Más específicamente, en esta 
fase se usan una serie de reglas de razonamiento deductivo. Estas reglas, a 
partir del conocimiento simplificado de la situación del alumno obtenido en 
la fase previa, califican la adecuación de cada asignatura al alumno de un 
modo que puede no ser totalmente preciso (puesto que se han omitido detalles 
para conseguir un método de resolución más eficiente). Por otro lado, se 
mantiene un registro con las reglas que se han podido aplicar para cada 
asignatura: este registro constituye un resumen de los motivos que llevan a 
la calificación final. En esta fase, pues, se incluye el subproblema de 
evaluación del grado de recomendación.

\item El refinamiento y la adaptación consisten en pasar de una solución 
abstracta a una solución concreta. De este modo, podremos pasar de una 
solución con distintas recomendaciones, que nos encajaría con el mismo 
tipo de estudiante, a nuestro alumno. En nuestro caso particular, escogeremos 
hasta un máximo de seis asignaturas entre todas las recomendadas por el 
sistema según los datos concretos del alumno para poder depurar mejor la 
recomendación. Concretamente, en esta fase se seleccionan las asignaturas que 
se mostrarán al usuario junto a los motivos por los cuales se recomiendan. 
Además, se garantiza (esta vez teniendo en cuenta todo el conocimiento 
concreto que se tiene del alumno) que la recomendación final cumple todas las 
restricciones requeridas. Esta fase, pues, corresponde al último de los 
subproblemas establecidos.

\end{enumerate}

A continuación, detallamos el proceso de razonamiento seguido por el sistema 
basado en el conocimiento para resolver cada uno de los subproblemas 
explicados en la \autoref{sec:subproblemas}. Es decir, ofrecemos una 
descripción a alto nivel de las reglas de deducción que permiten al sistema 
encontrar una solución.

En el primer subproblema, la obtención de datos del usuario, no hay ningún
tipo de razonamiento propiamente. Esta parte del sistema se limita a mostrar 
al usuario un conjunto de preguntas sobre sus preferencias y restricciones, 
ofreciendo la información necesaria sobre el formato de las respuestas 
esperado, y a almacenar apropiadamente el conocimiento obtenido a partir de 
las respuestas del usuario.

El segundo subproblema trata de inferir, a partir del expediente, 
todo aquel conocimiento acerca de las preferencias del alumno que no ha sido 
proporcionado explícitamente por este. Notamos que solo se infieren 
preferencias, puesto que no es posible conocer con certeza si las tendencias 
observadas en el expediente son debidas a alguna restricción estricta o no, 
si bien se puede asumir que es mejor que estas tendencias sigan cumpliéndose 
dentro de lo posible y, por ello, se expresan en forma de preferencias del 
alumno, con lo cual no se eliminan posibles soluciones. Es decir, este 
conocimiento inferido simplemente guiará la recomendación hacia soluciones 
parecidas a las matrículas del alumno en los cuatrimestres anteriores.

En particular, se infiere que es preferible restringir el número máximo de 
asignaturas a matricular, la cantidad máxima de horas de dedicación y la 
cantidad máxima de horas destinadas a prácticas de laboratorio a las medias 
por cuatrimestre de asignaturas matriculadas, cargas de trabajo (en horas) 
asumidas y prácticas de laboratorio, respectivamente. Si el último 
cuatrimestre el alumno cursó todas las asignaturas en un mismo horario, se 
asume que ese es su horario preferido. Asimismo, si el alumno ya está cursando
una especialidad del grado y, de hecho, ya le quedan muy pocas asignaturas 
para completarla, se infiere que es prioritario terminarla.
En cuanto a las competencias transversales, se asume que el alumno da 
prioridad a aquellas que se desarrollan en las asignaturas que ha cursado 
hasta el momento. Del mismo modo, se infiere que el 
alumno está interesado en aquellos temas especializados tratados en las 
asignaturas que ha cursado previamente. 
Finalmente, un análisis más detallado del expediente del alumno permite 
estimar la capacidad de este para asumir asignaturas más o menos difíciles. 
Concretamente, se comprueba si el alumno ha aprobado a la primera todas las 
asignaturas consideradas fáciles (donde se determina que una asignatura es 
fácil si la gran mayoría de la gente que se matricula la aprueba) y, en ese 
caso, si ha obtenido una calificación de al menos cuatro puntos en las 
asignaturas difíciles cursadas. 

El tercer subproblema consiste en simplificar los datos de los que disponemos 
mediante descripciones cualitativas que permitirán realizar la recomendación 
de forma más eficiente y sencilla, puesto que se reduce el número de factores 
a tener en cuenta. Es decir, para cada restricción o preferencia disponible, 
se evalúan los parámetros que la caracterizan y, según unos rangos, se 
sustituyen por valores más genéricos (con un dominio mucho más reducido). 

En particular, el volumen de trabajo asumible se califica como \textit{alto}, 
\textit{medio} o \textit{bajo} según si el número máximo de asignaturas a 
matricular es superior a cinco, entre tres y cuatro o como mucho dos, 
respectivamente. Similarmente, el número de horas de dedicación semanales se 
divide en \textit{alto}, \textit{medio} y \textit{bajo}. 
El resto de conocimiento del problema, como por ejemplo la dificultad 
asumible, la especialidad que se cursa y si es importante completarla, la 
disponibilidad horaria o los temas de interés, no se abstrae: se trata de 
parámetros que no se pueden agrupar de forma más general y que son necesarios 
de esta forma para el proceso de asociación posterior.

En el subproblema de asociación, esencialmente se comprueban algunas 
restricciones por asignatura y se calcula qué preferencias satisface cada 
asignatura. Esto permite, de entre las asignaturas que cumplen todas las 
restricciones satisfechas, establecer un orden de prioridad en la 
recomendación según la cantidad de preferencias cumplidas. Además, estas 
preferencias conforman los motivos que se ofrecerán al alumno para la 
recomendación. Sin embargo, algunas de las restricciones conciernen a todo 
el conjunto de asignaturas recomendadas (por ejemplo, el número máximo de 
asignaturas a matricular) y, por ello, no se tratan en esta fase, sino que 
se aseguran en el refinamiento final.

Concretamente, para cada asignatura se registra si está disponible en el 
horario preferido, si tiene la dificultad adecuada, si es de la especialidad 
que el alumno desea completar (en caso de que este desee completar la 
especialidad cursada), si trata algunos de los temas de interés del alumno 
o si desarrolla alguna de las competencias transversales preferidas. Se 
calcula también cuántas restricciones satisface cada asignatura para la criba 
de aquellas que no las cumplen todas. Asimismo, entre las restricciones 
comprobadas en esta fase, se analiza qué asignaturas están disponibles para 
el alumno; es decir, cuáles son las asignaturas que aún no ha cursado y 
que está en disposición de cursar porque ya ha completado el número de 
créditos necesarios (por ejemplo, un alumno no puede empezar la especialidad 
ni matricularse de asignaturas optativas hasta que no ha completado una 
cantidad mínima de créditos) y cumple todos sus requisitos, precorrequisitos 
y orrequisitos (se trata, de hecho, de un tipo de restricciones implícitas 
dadas por el plan de estudios).

El último subproblema consiste en generar la recomendación final y mostrarla 
al usuario. En este paso, además, hay que refinar las soluciones que se 
obtendrían directamente del subproblema anterior, puesto que algunas 
restricciones y preferencias no han podido ser comprobadas correctamente 
con el conocimiento simplificado de la abstracción o bien conciernen a todo 
el conjunto de asignaturas a recomendar. 

En esta parte, pues, se tratan las restricciones y preferencias relacionadas 
con la carga de trabajo permitida (puesto que es necesario conocer los valores 
exactos introducidos por el alumno): la cantidad máxima de asignaturas a 
matricular, el número de horas de dedicación semanales y el número de horas 
en prácticas de laboratorio. En esta parte también se comprueba la compleción 
de la especialidad (si procede) y el cumplimiento de los correquisitos de las 
asignaturas recomendadas. Teniendo en cuenta toda esta información concreta, 
se genera la recomendación final, constituida por un máximo de seis 
asignaturas que se adecúan en la medida de lo posible a las preferencias del 
alumno. Es decir, se intenta que se cumpla el máximo número posible de 
preferencias. Además, se asegura la satisfacción de todas las restricciones 
en esta recomendación. En caso de que sea imposible ofrecer una recomendación 
con todas las restricciones impuestas, se advierte al alumno de ello.




