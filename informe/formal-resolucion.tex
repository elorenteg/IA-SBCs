
\subsection{Razonamiento para la resolución} \label{sec:razonamiento}

Una vez analizados informalmente todos los elementos que constituyen el 
problema y establecida una representación formal de todo el conocimiento 
necesario sobre el dominio (es decir, la ontología), se puede justificar 
formalmente que el método de resolución señalado es adecuado y detallarlo 
aún más.

Como se ha indicado en la \autoref{sec:viabilidad}, el problema al que nos 
enfrentamos es un problema de análisis, en el que hay que escoger una solución
de entre un conjunto finito: en particular, hay que seleccionar el subconjunto 
de asignaturas más adecuado para el alumno en función de sus preferencias y 
restricciones. Para ello, pues, el sistema debe evaluar la adecuación de 
cada asignatura a un estudiante teniendo en cuenta el conocimiento del que 
dispone sobre las asignaturas, el plan de estudios y el alumno, y así 
clasificar las asignaturas en recomendables o no recomendables (es decir, hay 
que interpretar los datos de entrada del problema para poder seleccionar una 
solución adecuada; esta es precisamente una caracterización de los problemas 
de análisis).

Además, partiendo de unos datos de entrada que proporcionan un conocimiento 
incompleto hay que asociar a estos una solución mediante razonamientos 
heurísticos. En conclusión, el método más adecuado para resolver este problema 
es la clasificación heurística. Para nuestro problema de recomendación la resolución consistirá en tres fases:

\begin{enumerate}
\item La \textbf{abstracción de datos} cosiste en pasar de un problema concreto, la situación de nuestro alumno, y pasarlo a un problema abstracto, una generalización para tratar a todos los alumnos. Si el alumno nos ha introducido preferencias y restricciones, la abstracción será sencilla porque ya tendremos toda la información que necesitaremos para el sistema. En caso contrario, deberemos deducir sus preferencias según su expediente.
\item La \textbf{asociación herurística} consiste en pasar de un problema abstracto a una solución abstracta. En esta fase se quiere obtener una posible solución que normalmente ha funcionado para un mismo patrón de estudiantes y del que se genera una serie de recomentaciones de asingaturas según si son altamente recomendables o solo recomendables.
\item El \textbf{Refinamiento} y la \textbf{adaptación} consiste en pasar de una solución abstracta a una solución concreta. De este modo, podremos pasar de una solución con distintas recomendaciones, que nos encajaría con el mismo tipo de estudiantes, a nuestro alumno. En esta fase, escogeremos hasta un máximo de seis asignaturas entre todas las recomendadas por el sistema según los datos concretos del alumno para poder depurar mejor la recomendación.
\end{enumerate}
