
\subsection{Razonamiento para la resolución} \label{sec:razonamiento}

Una vez analizados informalmente todos los elementos que constituyen el 
problema y establecida una representación formal de todo el conocimiento 
necesario sobre el dominio (es decir, la ontología), se puede justificar 
formalmente que el método de resolución señalado es adecuado y detallarlo 
aún más.

Como se ha indicado en la \autoref{sec:viabilidad}, el problema al que nos 
enfrentamos es un problema de análisis, en el que hay que escoger una solución
de entre un conjunto finito: en particular, hay que seleccionar el subconjunto 
de asignaturas más adecuado para el alumno en función de sus preferencias y 
restricciones. Para ello, pues, el sistema debe evaluar la adecuación de 
cada asignatura a un estudiante teniendo en cuenta el conocimiento del que 
dispone sobre las asignaturas, el plan de estudios y el alumno, y así 
clasificar las asignaturas en recomendables o no recomendables (es decir, hay 
que interpretar los datos de entrada del problema para poder seleccionar una 
solución adecuada; esta es precisamente una caracterización de los problemas 
de análisis).

Además, partiendo de unos datos de entrada que proporcionan un conocimiento 
incompleto hay que asociar a estos una solución mediante razonamientos 
heurísticos. En conclusión, el método más adecuado para resolver este problema 
es la clasificación heurística. 

%%% TODO : Fases de la classificació heurística - subproblemes. %%%



