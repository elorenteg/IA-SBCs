
\subsection{Viabilidad del desarrollo} \label{sec:viabilidad}

De la descripción hecha en la \autoref{sec:descripcion}, se deduce que el 
problema tratado es esencialmente un problema de análisis: medianto el 
conocimiento del que se dispone sobre las asignaturas y el estudiante, se 
espera que el sistema evalúe su compatibilidad y, a partir de esta evaluación, 
clasifique las asignaturas en recomendables (con ciertos grados) o no 
recomendables.

Por lo tanto, este problema no dispone de una solución algorítmica directa 
usando los métodos clásicos, puesto que se trata de un problema complejo y 
altamente dependiente del conocimiento sobre sus múltiples elementos 
(esencialmente, las particularidades asociadas a las asignaturas y al plan de 
estudios de la FIB y las necesidades y preferencias de los estudiantes). 
Además, este conocimiento es inicialmente incompleto (por ejemplo, un 
estudiante no tiene por qué rellenar todos los posibles campos a tener en 
cuenta en la recomendación) y no hay un método exacto para establecer los 
grados de recomendación de las asignaturas, sino que hay que usar métodos 
heurísticos que aproximen los razonamientos lógicos que podría hacer una 
persona (y sin disponer de un conocimiento completo).

Por ende, parece una elección razonable modelar este problema como un problema 
de clasificación heurística, que se puede resolver con un sistema basado en el 
conocimiento. Este puede seguir un esquema basado en la abstracción de los 
datos relevantes a partir de toda la información disponible para la creación 
posterior de una cierta asociación de asignaturas al alumno con métodos 
heurísticos de razonamiento, determinando así la recomendación final.

Si bien es cierto que no se dispone de todo el conocimiento del problema, el 
sistema almacena los datos de los elementos principales de la solución: las 
asignaturas y los alumnos. Concretamente, hay toda una serie de 
características y relaciones que describen (en algunos casos cualitativamente) 
las asignaturas y las preferencias y necesidades de los alumnos, permitiendo 
así razonar sobre estos para obtener un cierto conocimiento (aunque este puede 
no ser del todo preciso ante la falta de datos en la entrada) para generar 
una recomendación que se ajuste el máximo posible al alumno según la 
información de la que se dispone.

Los razonamientos que llevan a la recomendación de asignaturas según los 
criterios establecidos se pueden emular usando las reglas de producción de 
un sistema basado en el conocimiento.

En conclusión, la implementación de un sistema basado en el conocimiento para 
la resolución de este problema parece factible (puesto que se dispone de todos 
los elementos necesarios para ello) y, por otra parte, no se conocen 
algoritmos en el sentido ``clásico'' para tratarlo de forma exacta.


