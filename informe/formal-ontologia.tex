
\subsection{Ontología del dominio} \label{sec:ontologia}

El conocimiento adquirido para este problema se representa mediante una 
ontología. En esta aparecen todos los conceptos detallados en la 
\autoref{sec:conceptualizacion} y las relaciones entre ellos, puesto que debe 
permitir al sistema basado en el conocimiento razonar sobre ellos de forma 
adecuada. Para formalizarlos deberemos pensar en una forma para representarlos 
y que nuestro Sistema Basado en el Conocimiento lo entienda.

La ontología empieza con el Alumno que se podrá distinguir según su DNI. 
Desde él podremos acceder a todo su expediente de las asignaturas a las 
que se ha presentado y a sus preferencias sobre como le gustaría ser su 
próxima elección de matrícula. Ésto nos sirve para que desde la clase Alumno 
podamos gestionar sus preferencias y restricciones o inferirlas si es 
necesario y así poder escoger las mejores asignaturas para una buena recomendación.

El expediente único del alumno tendrá todas las convocatorias de examenes 
del alumno. Para distinguir una convocatoria usaremos la asignatura y el 
cuatrimestre en el que se ha presentado. Ésto nos permitirá que un alumno 
se pueda presentar a la misma asignatura en más de una ocasión si éste la 
suspendiera. Como atributo tendremos la calificación y su horario.

Las asignaturas son los conceptos que permanecerán estáticos durante toda 
la ejecución del sistema. Su conocimiento no podrá ser modificado por el 
alumno porque es independiente a él y es algo nos proporciona la Facultad. 
\textbf{(acabar de explicar)}

Para poder tener acceso a las preferencias y restricciones hemos decidido 
agruparlos en un solo concepto porque la información que debemos obtener es 
la misma para ambas, pero con un matiz distinto. Para distinquir entre cada 
uno tendremos un atributo booleano esPreferencia que  nos indica si se ha 
de cumplir siempre (restricción) o si no es algo primordial el que se cumpla 
(preferencia). Heredado de él tendremos un subconcepto para cada 
preferencia/restricción que se quiera introducir en el sistema. Para poder 
diferenciarlos, usaremos el atributo booleano y su subconcepto. De este modo 
podremos tener una restricción y una preferencia de un mismo subconcepto.

A continuación mostramos el esquema de la jerarquía de la ontología 
anteriormente descrita.

\textbf{(grafo de jerarquía)}