
\subsection{Elementos del problema} \label{sec:elementos}

Tal y como se ha explicado en la \autoref{sec:descripcion}, los elementos de 
este problema se pueden clasificar en dos grandes grupos: por un lado están 
las asignaturas, sus características particulares y las relaciones entre ellas 
y con el plan de estudios; por otro lado hay los alumnos, sus expedientes y 
sus restricciones y preferencias en cuanto a las asignaturas que pueden 
cursar.

Las asignaturas se identifican con las siglas de su nombre. Para cada 
asignatura del grado, hay que tener una serie de datos sobre su lugar en el 
plan de estudios y la cantidad de trabajo asociada, así como información 
acerca de su contenido, para poder satisfacer las restricciones que pueda 
imponer el alumno y tener en cuenta en la medida de lo posible sus 
preferencias. 

Así, se almacena para cada asignatura el curso en el que está planeada y el 
tipo de asignatura de que se trata: obligatoria, de especialidad u optativa. 
En el caso de las asignaturas de especialidad, se sabe también a qué 
especialidades pertenecen. Con estos datos, se puede comprobar la adecuación 
de la asignatura al alumno según su progreso en el plan de estudios 
(por ejemplo, es importante saber si un alumno todavía tiene pendientes 
algunas asignaturas obligatorias o si está en una determinada especialidad a 
la hora de recomendarle asignaturas). Como el plan de estudios obliga a los 
alumnos a cursar por lo menos una asignatura con el nivel tres de cada una de 
las competencias transversales, se tiene en cuenta qué competencias 
transversales y a qué nivel se desarrollan en cada asignatura: con esta 
información, se puede ofrecer una recomendación que permita desarrollar las 
competencias restantes a un alumno que esté en una fase avanzada del grado.
Se considera también el número de créditos a los que corresponde cada 
asignatura y la distribución de esta carga de trabajo en horas de teoría, de 
problemas o de laboratorio, así como el número de matriculados y el 
porcentaje de aprobados del cuatrimestre anterior. A partir de estos datos, 
se puede hacer una estimación de la dificultad y el esfuerzo que conllevan 
las asignaturas para ofrecer una recomendación razonable según las 
capacidades y limitaciones del alumno en cuestión. Otro dato importante a 
tener en cuenta es la disponibilidad horaria de las asignaturas (algunas se 
ofrecen solo en horario de mañana o de tarde), puesto que habrá que respetar 
las restricciones horarias del alumno.

Hay que tener en cuenta asimismo el contenido de las asignaturas para hacer 
una recomendación adecuada a los gustos e intereses del alumno. Para ello, 
se tiene una clasificación de temas genéricos, especializados y no 
informáticos. Algunos temas especializados están interrelacionados según su 
afinidad. De este modo, se conocen los temas que trata cada asignatura y se 
puede utilizar esta información, junto a los intereses mostrados por un 
alumno, para ofrecerle una recomendación de asignaturas personalizada (es 
decir, se intentará recomendar asignaturas que traten los temas en los que 
el alumno está interesado o bien temas afines a estos). 

%%% TODO : Alumnos - expediente, preferencias y restricciones %%%

%%% TODO %%%


