
\subsection{Objetivos del sistema} \label{sec:objetivos}

Definimos finalmente los objetivos clave que debe satisfacer el sistema para 
resolver correctamente el problema que nos planteamos. 

En primer lugar, el sistema debe ser capaz de obtener las restricciones y 
preferencias respecto a las asignaturas del alumno mediante la formulación de 
preguntas a este. Sin embargo, el usuario debe poder ignorar estas preguntas 
y, bajo estas circunstancias, el sistema tiene que ser capaz de inferir 
algunas restricciones y preferencias del alumno tras un análisis de su 
expediente. 

El objetivo fundamental del programa es, una vez recopilado este conocimiento 
sobre el alumno, y sumado al conocimiento que ya tiene que haber almacenado en 
el sistema sobre las características particulares de las asignaturas y del 
plan de estudios, establecer una valoración de la adecuación de las 
asignaturas al perfil del alumno (es decir, teniendo en cuenta sus intereses, 
preferencias y restricciones). Basándose en esta valoración, el programa 
tiene que hacer una selección final de hasta un máximo de seis asignaturas 
(aquellas que se han determinado ``más recomendables'') para ofrecer la 
recomendación final al alumno.

Finalmente, el sistema tiene que presentar al alumno la recomendación de 
asignaturas de forma clara. En particular, el sistema debe ser capaz de 
mostrar, para hasta un máximo de seis asignaturas recomendadas, un grado 
cualitativo de recomendación (altamente recomendable o solo recomendable) y 
los motivos que han conducido al sistema de razonamiento a determinar la 
recomendación de esa asignatura (es decir, una justificación del porqué de la 
recomendación). Evidentemente, estas asignaturas tienen que ser válidas para 
que el alumno las matricule el siguiente cuadrimestre y tienen que cumplir 
todas las restricciones impuestas.


