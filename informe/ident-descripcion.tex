
\subsection{Descripción del problema} \label{sec:descripcion}

El proceso de matriculación de asignaturas en la FIB por parte de los 
estudiantes requiere de la consideración de una multitud de factores muy 
variados e interdependientes. La gran diversidad de asignaturas con sus 
distintos requisitos, horarios y grupos puede plantear un reto a la hora de 
elegir un subconjunto de ellas adecuado a la situación y a la predilección 
por algunos temas del alumno. Con el objetivo de simplificar al máximo este 
trámite, se plantea el desarrollo de una aplicación de recomendación de 
asignaturas que tenga en cuenta todas estas variables.

El sistema empleado para hacer la recomendación se basa en las características 
de las asignaturas. Además, el sistema dispone de los expedientes de los 
alumnos de la FIB y se sirve de datos adicionales recopilados mediante 
preguntas a los usuarios (estudiantes) para aumentar la base de datos sobre 
estos. A partir de esta información, pues, se pueden construir recomendaciones 
personalizadas más ajustadas a la realidad individual del estudiante.

En particular, para cada asignatura de la FIB se almacena su nombre, el curso 
en el que está ubicada según el plan de estudios, el número de créditos a los 
que corresponde, su distribución en clases de teoría, problemas y laboratorio, 
si es de proyecto, si es obligatoria, de especialidad (y, en este caso, a qué 
especialidad corresponde) u optativa, qué temas trata, las competencias 
transversales asociadas a esta, los horarios en los que está disponible, el 
número de matriculados y de aprobados del cuatrimestre anterior, ...

Se dispone además de conjuntos de temas generales, especializados y no 
informáticos que engloban algunas asignaturas. De entre estos, los temas 
especializados se relacionan entre sí según un grado de afinidad. Esta 
clasificación de las asignaturas en temas permite ofrecer recomendaciones 
adecuadas a los intereses de los alumnos.

El sistema contiene también los expedientes académicos de los alumnos, 
estructurados según las convocatorias a las que se han presentado de cada 
asignatura: para cada una de estas, se almacena el cuadrimestre y el horario 
en el que se cursó la asignatura y la calificación final obtenida.

Además de la información contenida en el sistema, se ofrece la posibilidad de 
que el alumno añada información adicional sobre sus intereses y restricciones 
que conciernen a la elección de las asignaturas. En particular, este puede 
especificar un número máximo de asignaturas a mostrar, un máximo de horas de 
dedicación esperadas o de dedicación a prácticas esperadas, los horarios en 
los que puede asistir a clase, los temas de su interés, un interés en terminar 
la especialidad o en adquirir determinadas competencias transversales, un 
nivel de dificultad aceptable para las asignaturas recomendadas, ...

Sin embargo, el alumno no tiene la obligación de introducir todos estos datos. 
En tal caso, algunas de estas preferencias y restricciones son inferidas por 
el sistema a partir de los datos de los que se dispone: su expediente hasta 
el momento y los datos sobre las asignaturas y el sistema docente de la FIB. 
Es decir, en ausencia de información explícitamente proporcionada por el
propio alumno, el sistema toma decisiones sobre sus preferencias y 
restricciones basándose en la información previa (o sea, aquella proveniente 
de un análisis del expediente del alumno) y del conocimiento sobre el dominio 
(o sea, sobre las asignaturas disponibles y sus características y relaciones 
con los distintos elementos del plan de estudios de la FIB).

El conocimiento inferido se estructura según unos parámetros que permiten 
relacionar un alumno con distintas asignaturas para determinar la 
recomendación final: su disponibilidad horaria, la dedicación posible, el 
volumen de trabajo que puede asumir, la lista de temas de su interés, la 
especialidad que cursa, su interés por terminarla, ...

A partir de los parámetros descritos, es posible realizar una búsqueda sobre 
las asignaturas disponibles y asociar a cada una de ellas un grado de afinidad 
con el alumno para la recomendación (es decir, un índice que estima cuán 
apropiada es cada asignatura para el alumno). Finalmente, una vez determinados 
estos, se escogen hasta un máximo de seis asignaturas (aquellas que el sistema 
ha señalado más adecuadas) para elaborar la recomendación final, que incluye 
asimismo un grado de recomendación para cada una de ellas y los motivos que 
han conducido al sistema a recomendar esas asignaturas.


