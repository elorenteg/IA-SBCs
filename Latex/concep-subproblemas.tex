
\subsection{Estructura del problema} \label{sec:subproblemas}

A fin de resolver este problema, en el que intervienen muchos elementos 
altamente diferenciados, es importante descomponerlo en subproblemas más 
simples que faciliten un tratamiento sistemático mediante razonamientos 
sobre el conocimiento adquirido. 

La descripción de los diversos elementos del problema en la 
\autoref{sec:elementos} ya sugiere una cierta estructura en subproblemas 
diferenciados que se pueden resolver de forma individual. 

Primeramente, hay que obtener el máximo de información posible del usuario 
(es decir, del alumno para el que hay que generar una recomendación de 
asignaturas). Es decir, este subproblema consiste en consultar al usuario 
sobre sus preferencias y las restricciones que quiera imponer a la 
recomendación mediante un sistema de preguntas directas que se adapte con 
nuevas preguntas en función de las respuestas obtenidas (de modo que el 
usuario pueda ofrecer suficiente conocimiento útil respondiendo al mínimo 
número de preguntas posible).

En segundo lugar, se debe obtener el conocimiento faltante a partir del 
expediente del alumno. Es decir, hay que inferir aquellas preferencias y 
restricciones que el usuario ha omitido mediante un sistema de reglas de 
razonamiento basadas en su expediente pasado. Con estas dos fases, se 
determinan completamente las restricciones que deberán cumplir las asignaturas 
recomendadas y las preferencias que guiarán el proceso de selección de 
asignaturas. 

El tercer subproblema consiste en abstraer de las restricciones y 
preferencias del alumno obtenidas un conjunto de características clave 
suficientemente descriptivas y que vengan determinadas por unos valores mucho 
más restringidos. Esto facilitará la calificación de la adecuación de las 
asignaturas al alumno (puesto que el tamaño de los datos a analizar será 
menor).

Tras esta abstracción, hay que asociar a cada asignatura disponible para la 
matrícula del alumno un cierto grado de adecuación para la recomendación, 
basándose en las características del alumno y las de la asignatura. Es decir, 
el cuarto subproblema consiste en determinar una estimación de cuán 
recomendables son las asignaturas usando el conocimiento abstraído 
anteriormente (que no será el conocimiento completo sobre el alumno, sino una 
simplificación de este). De hecho, en el proceso hay que recordar las razones 
en las que se basa esta estimación.

Finalmente, una vez calculados estos niveles de recomendación, se puede 
elaborar la solución final seleccionando aquellas asignaturas más idóneas 
(hasta un máximo de seis). En este subproblema, puede ser necesario comprobar 
que las asignaturas elegidas cumplen en efecto todas las restricciones 
impuestas (puesto que la recomendación hecha en el subproblema anterior puede 
haber ignorado algunos detalles y, además, algunas restricciones y 
preferencias pueden afectar al conjunto total de asignaturas recomendadas).





