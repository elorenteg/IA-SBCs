
\subsection{Algunos ejemplos del conocimiento experto} \label{sec:ejemplos}

En las secciones previas, se ha hablado de la adecuación de las asignaturas a 
los alumnos de forma abstracta. Sin embargo, no se han establecido de ningún 
modo los criterios de valoración de las asignaturas que el sistema utilizará 
para generar una recomendación. A continuación, procedemos a ofrecer algunos 
ejemplos concretos del conocimiento experto del sistema que permite efectuar 
esta valoración, a parte del conocimiento objetivo sobre el plan de estudios 
y el expediente del alumno ya descrito.

En el problema de inferencia de conocimiento a partir del expediente, se usa 
implícitamente una gran cantidad de conocimiento sobre el dominio para obtener 
conclusiones. Por ejemplo, el sistema asume que, si un alumno ha realizado 
todas las asignaturas del cuatrimestre previo en un mismo horario, es porque 
tiene preferencia por este horario. Del mismo modo, el sistema deduce que la 
carga de trabajo o la dificultad de las asignaturas soportada por el alumno 
es similar a la de los cuatrimestres previos (por lo tanto, si el alumno ha 
matriculado pocas asignaturas en los cuatrimestres anteriores, es razonable 
ofrecer una recomendación de pocas asignaturas a matricular). También se 
deduce que los temas de interés del alumno son aquellos que se han tratado en 
las asignaturas matriculadas anteriormente (es decir, es razonable pensar que 
un alumno se matricula de aquellas asignaturas que le interesan).

En el problema de valoración de las asignaturas también se utiliza 
conocimiento experto sobre el dominio. Entre otros usos, el sistema determina 
la dificultad aproximada de las asignaturas en función del porcentaje de 
aprobados y de la nota media del último cuatrimestre (parece razonable admitir 
que una asignatura es fácil si la gran mayoría de los matriculados la aprueban 
y si la nota media es alta y que es difícil en caso contrario). 



