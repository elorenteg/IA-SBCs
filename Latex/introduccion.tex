
\section{Introducción} \label{sec:intro}

Algunos problemas no se pueden resolver satisfactoriamente con algoritmos de
carácter general, sino que su resolución requiere de conocimiento específico 
sobre elementos del dominio de definición del problema que permita tomar 
decisiones adecuadas. Por ello, la inteligencia artificial hace uso de 
sistemas basados en el conocimiento para solucionar problemas de alta 
complejidad.

En este trabajo, lo ejemplificamos mediante la implementación de un sistema 
de recomendación de asignaturas de la FIB a sus alumnos teniendo en cuenta sus 
preferencias y las restricciones sobre las asignaturas de la FIB, entre otros 
criterios. Esta tarea es altamente dependiente de la información disponible 
sobre la estructura de las asignaturas y sobre las características propias de 
los alumnos. Se trata, pues, de un problema para el cual tiene mucho sentido 
el uso de un sistema basado en el conocimiento, separando así el tratamiento 
de la información subyacente de los algoritmos de razonamiento que generan 
las recomendaciones.

Se desarrolla, pues, una ontología que almacena la información sobre las 
asignaturas y los estudiantes usando \texttt{Protégé} y un sistema de reglas 
que describen el proceso de toma de decisiones usando \texttt{CLIPS}. El 
desarrollo de la práctica, así como la estructura de este informe, siguen un 
modelo esencialmente en cascada basado en la ingeniería del conocimiento y 
que se divide en las siguientes fases: identificación del problema, 
conceptualización, formalización, implementación y prueba. A pesar de la 
dimensión reducida de esta práctica, sin embargo, hemos iterado algunas veces 
sobre este esquema, volviendo a fases anteriores para mejorar el diseño de 
nuestra solución.

En el resto de este documento, explicamos los pasos del desarrollo de nuestra 
solución y mostramos y analizamos los resultados obtenidos.

\clearpage


