
\subsection{Fuentes de conocimiento disponibles} \label{sec:conocimiento}

En todo momento se ha asumido que el sistema basado en el conocimiento 
diseñado para la resolución de nuestro problema dispone del conocimiento 
necesario sobre las asignaturas y los alumnos. Para que esto sea cierto, pues, 
hace falta considerar las fuentes de conocimiento disponibles y analizar si 
son suficientes para obtener el conocimiento descrito.

La primera y más importante fuente de conocimiento es la FIB. Como institución 
encargada de ello, la FIB dispone de toda la información relativa a las 
asignaturas y a los planes de estudios, así como del expediente académico de 
sus alumnos. Se trata, pues, de una fuente fiable que ya almacena la mayor 
parte del conocimiento requerido de forma estructurada e, incluso, mantiene 
parte de ello en sitios públicos (básicamente, su página web). 

Observamos, aún así, que en el caso de esta práctica no vamos a disponer de 
los expedientes reales de los alumnos de la facultad. Por lo tanto, para la 
realización de la práctica nos inventaremos algunos ejemplos de distintos 
tipos de expedientes para probar el funcionamiento del sistema. Sin embargo, 
en un caso real sí podríamos tener acceso a estos datos a través de la FIB.

La otra fuente de conocimiento importante son los propios alumnos. Estos son 
los únicos conocedores reales de sus circunstancias: necesidades y 
preferencias relativas a las asignaturas que pueden cursar. Estos datos son 
vitales para generar recomendaciones ajustadas a su perfil. A pesar de todo, 
es posible que no proporcionen toda la información al sistema, en cuyo caso 
se infieren algunos datos razonables a partir de la información proporcionada 
por la FIB.

En definitiva, estas son las fuentes más fiables de las que podemos obtener 
todo el conocimiento necesario para nuestro sistema.



