
\subsection{Segundo juego de pruebas} \label{sec:prueba-2}

Este segundo caso de prueba se centra en una persona que ha superado todas 
las asignaturas del primer curso siguiendo el plan de estudios propuesto por 
la FIB. Se trata también de una persona con dedicación completa a sus estudios
universitarios y que, por lo tanto, no impone restricciones sobre la 
recomendación ni tiene ninguna preferencia particular.

Por lo tanto, parece que lo más razonable sería que el sistema recomendase 
que el alumno se matriculara de todas las asignaturas del tercer cuatrimestre 
según el plan de estudios (ya que el alumno parece tener capacidad y 
disponibilidad suficiente para asumir esta carga de trabajo y aún no se 
encuentra en disposición de elegir asignaturas de especialidad u optativas).

El objetivo de este segundo juego de pruebas es, pues, determinar si el 
sistema es capaz de ofrecer la recomendación ``obvia'' en un caso simple, 
teniendo en cuenta su conocimiento sobre la estructura del plan de estudios 
propuesto por la FIB.

\lstinputlisting[caption={Salida original del segundo juego de pruebas.},%
        label={lst:prueba-2}]%
    {resultado-2.txt}

El \autoref{lst:prueba-2} muestra la salida original del programa con este 
caso particular. Otra vez, la recomendación obtenida se asemeja mucho a lo 
esperado desde nuestro punto de vista de expertos en el dominio. 
Concretamente, se recomiendan cuatro de las cinco asignaturas del tercer 
cuatrimestre del grado. En particular, de estas, se recomiendan las más 
parecidas a las del primer curso, es decir, las continuaciones de estas 
(puesto que es mejor hacerlas cuanto antes, cuando los conocimientos todavía 
son recientes para aumentarlos y reforzarlos).

La única diferencia es que solamente se han recomendado cuatro asignaturas (y 
no las cinco disponibles en este cuatrimestre) debido al proceso de inferencia:
como el alumno no ha especificado un número máximo de asignaturas a cursar, 
el sistema ha deducido que el alumno prefiere cursar cuatro asignaturas por 
cuatrimestre, puesto que es precisamente lo que ha hecho en los dos 
cuatrimestres previos. Se podría discutir que, en este caso, sería mejor 
recomendar las cinco asignaturas (ya que las asignaturas de segundo curso 
tienen menos créditos), pero como el alumno no ha sacado notas muy altas hasta 
el momento (a pesar de haber aprobado todas las asignaturas), el sistema 
concluye que es mejor recomendar esas cuatro asignaturas.

En definitiva, en este caso también se ha obtenido una recomendación adecuada 
a la situación del alumno; de hecho, la recomendación del sistema es 
prácticamente la misma que hubiésemos ofrecido como expertos en el dominio.


