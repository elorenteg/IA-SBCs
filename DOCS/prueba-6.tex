
\subsection{Sexto juego de pruebas} \label{sec:prueba-6}

Este último sujeto de prueba es una persona en su último curso del grado. 
Esta persona ha completado todas las asignaturas obligatorias y todas las de 
la especialidad de computación siguiendo el plan de estudios propuesto por 
la FIB y, además, obteniendo calificaciones bastante buenas. Por ello, este 
alumno desea que se le recomienden asignaturas optativas acordes a sus gustos 
y que desarrollen las competencias transversales que todavía no ha obtenido. 
Además, este alumno ya empieza a pensar en su futuro fuera de la universidad 
y, por lo tanto, tiene algunas preocupaciones adicionales. 

Consideramos también dos versiones de este juego de pruebas. En la primera de 
ellas, el alumno está especialmente preocupado por su insuficiente 
conocimiento del inglés y el impacto que esto podría tener en su futuro 
laboral. Por este motivo, impone como restricción estricta que las asignaturas 
de las que quiere matricularse desarrollen la competencia transversal de la 
tercera lengua. Además, prefiere no tener una carga de trabajo excesiva para 
poder ir pensando en el trabajo de final de grado, así que impone algunas 
restricciones (no muy restrictivas) en cuanto a la dedicación a la 
universidad. Además de ello, el alumno muestra un cierto interés por algunos 
temas concretos. Por contra, en la segunda versión, el alumno se plantea la 
posibilidad de desarrollar el trabajo de final de grado en inglés y, por lo 
tanto, ya no es tan importante que las asignaturas trabajen la tercera lengua. 
En este caso, pues, la restricción sobre la competencia transversal pasa a 
ser una preferencia y ya no se imponen restricciones sobre la carga de trabajo
asumible.

Teniendo en cuenta la situación del alumno, se espera que se recomienden 
principalmente asignaturas optativas (o de alguna especialidad que se puedan 
cursar en concepto de optativas) sobre algunos de los temas que interesan al 
alumno y que desarrollen algunas de las competencias transversales preferidas, 
dando una especial importancia a la competencia de la tercera lengua en la 
primera versión.

Con este juego de pruebas, pues, se intenta confirmar el distinto 
comportamiento del sistema frente a entradas más o menos restrictivas, 
reforzando así los resultados observados en el juego de pruebas anterior. 
Además, este caso es sustancialmente distinto a los anteriores porque este 
alumno se encuentra ya cerca del final del plan de estudios y, por lo tanto, 
solo se pueden tener en cuenta asignaturas optativas, de otras especialidades 
o incluso alguna de la especialidad de computación que no se haya cursado. Es 
decir, el conjunto de asignaturas a tener en cuenta es mucho más reducido (a 
pesar de que no hay restricciones impuestas por el plan de estudios sobre 
requisitos o la necesidad de completar una especialidad).

\lstinputlisting[caption={Salida original de la primera versión del sexto %
        juego de pruebas.}, label={lst:prueba-6a}]%
    {resultado-6a.txt}

Los resultados obtenidos con la primera versión de este juego de pruebas se 
muestran en \autoref{lst:prueba-6a}. Se observa que, en este caso, el sistema 
ha dado mucha prioridad a las asignaturas que desarrollan la tercera lengua 
y ha sugerido una recomendación muy conservadora en este sentido. Así, las 
cuatro asignaturas recomendadas son optativas y tres de ellas están enfocadas 
básicamente a habilidades importantes para el mundo laboral y cursadas en 
inglés. Sin embargo, esto ha hecho que prácticamente se ignoren los temas de 
interés del alumno (solamente la asignatura de geometría computacional tiene 
en cuenta los intereses temáticos) y el resto de competencias transversales.
En definitiva, se trata de una recomendación más limitada de lo esperado en 
un principio, pero se puede entender por la importancia que da el sistema a 
las restricciones.

\lstinputlisting[caption={Salida original de la segunda versión del sexto % 
        juego de pruebas.}, label={lst:prueba-6b}]%
    {resultado-6b.txt}

El \autoref{lst:prueba-6b} muestra los resultados obtenidos con la versión 
menos restrictiva. En este caso, como era de esperar, se da una mayor 
importancia a los intereses temáticos del alumno y, en consecuencia, se 
recomiendan muchas asignaturas de la especialidad de computación basadas en 
los temas listados en las preferencias. En este caso, sin embargo, las 
competencias transversales preferidas han tenido un peso prácticamente nulo 
en la recomendación porque la mayoría de ellas ya se habían obtenido con 
las asignaturas cursadas previamente (y el sistema ha sido capaz de detectarlo 
a partir del expediente). Además, hay un cierto solapamiento entre las 
asignaturas recomendadas (tres de ellas se recomiendan por tratar el tema de 
tratamiento de datos). Esto se debe a las limitaciones del sistema a la hora 
de considerar los subconjuntos de asignaturas válidos para la recomendación 
(puesto que hay un gran número de ellos y, por lo tanto, el sistema aplica 
razonamientos heurísticos y voraces para obtener recomendaciones válidas en 
un tiempo factible a la práctica) y a su mayor facilidad para evaluar las 
asignaturas de forma individual. A pesar de todo, la recomendación presentada 
sigue siendo aceptable.

Con este juego de pruebas más complicado, hemos observado algunas limitaciones 
del sistema de recomendación. Así, en casos más complejos o con mayores 
restricciones, el sistema recomienda muchas asignaturas parecidas entre ellas 
y puede obviar algunas de las preferencias (porque se prioriza el número de 
preferencias satisfechas por asignatura). Aun así, es capaz de ofrecer 
recomendaciones aceptables, aunque no sean las mejores posibles desde el punto 
de vista de un experto.



