
\subsection{Primer juego de pruebas} \label{sec:prueba-1}

El primer sujeto de prueba es alguien que acaba de iniciar sus estudios en la 
FIB. Debido a la dificultad inicial para adaptarse a la universidad, esta 
persona ha suspendido la mitad de las asignaturas del primer cuatrimestre y 
busca una recomendación sobre las asignaturas que debería matricular para el 
segundo cuatrimestre. Esta persona se dedica exclusivamente a sus estudios 
universitarios, por lo cual no presenta ningún tipo de restricción ni 
preferencia explícita: prefiere que el sistema evalúe las opciones más 
convenientes. 

Dadas las circunstancias de esta persona, el sistema debería ofrecer una 
recomendación en la que se incluyan las dos asignaturas suspendidas del primer 
cuatrimestre (puesto que la opción más sensata es recuperarlas cuanto antes 
mejor, aprovechando los conocimientos que se hayan podido adquirir a pesar del 
suspenso y reforzándolos) y probablemente una o dos asignaturas del segundo 
cuatrimestre (de forma que se cumplan los requisitos del plan de estudios).

Así, con este juego de pruebas se pretende comprobar si el sistema razona 
correctamente dando prioridad a las asignaturas que ya se han matriculado 
previamente y se han suspendido.

\lstinputlisting[caption={Salida original del primer juego de pruebas.},%
        label={lst:prueba-1}]%
    {resultado-1.txt}

En el \autoref{lst:prueba-1} se muestra la salida original del programa con 
este ejemplo de prueba. Se observa que, en este caso tan simple, el sistema 
ofrece exactamente la recomendación esperada (e incluso es capaz de distinguir 
que las asignaturas repetidas son altamente recomendables, en contraposición 
con las del segundo cuatrimestre, que son solo recomendables). Se observa 
también que las asignaturas del segundo cuatrimestre recomendadas son las de 
matemáticas (porque el alumno ya ha aprobado la asignatura de matemáticas del 
primer cuatrimestre); es decir, las continuaciones naturales de las 
asignaturas suspendidas han tenido menos prioridad para el sistema de 
recomendación, como cabría esperar.

En conclusión, para este caso sencillo el sistema ofrece una recomendación 
que nos parece muy acertada, teniendo en cuenta nuestro conocimiento del 
dominio.


