
\section{Conclusión} \label{sec:conclusion}

En este trabajo hemos resuelto un problema práctico aplicando algunas de las 
metodologías de la ingeniería del conocimiento vistas en clase. A pesar de 
tratarse de un problema un poco simplificado respecto al problema que podría 
plantearse la FIB (que podría tener en cuenta más factores debido a la mayor 
disponibilidad de información del plan de estudios y del expediente de los 
alumnos), esta versión del problema ya nos ha permitido hacernos una idea del 
tipo de problemas a los que se pueden aplicar las técnicas de inteligencia 
artificial aprendidas.

En particular, hemos analizado con todo detalle el problema de recomendación 
de asignaturas a los alumnos y hemos desarrollado un sistema basado en el 
conocimiento capaz de resolverlo usando el conocimiento experto disponible 
sobre el dominio. Este sistema se compone esencialmente de una ontología 
desarrollada con \texttt{Protégé} en la que se representa el conocimiento del 
dominio y un programa en \texttt{CLIPS} que sintetiza el proceso de 
razonamiento seguido en la obtención de recomendaciones adecuadas mediante 
reglas de deducción lógica. Todo el desarrollo se ha llevado a cabo siguiendo 
las fases de la ingeniería del conocimiento: identificación del problema, 
conceptualización, formalización, implementación y prueba. Más precisamente, 
la resolución planteada se enmarca en las técnicas de asociación heurística.

De este modo, hemos podido comprobar como, en este tipo de problemas, si bien 
puede ser complicado o incluso imposible dar una solución perfecta usando las 
técnicas de programación más tradicionales, se puede aprovechar la mayor 
expresividad de un lenguaje de reglas de deducción y la potencia proporcionada 
por su motor de inferencia para expresar de forma relativamente simple un 
proceso de razonamiento que lleva a la obtención de soluciones razonablemente 
buenas.


\clearpage

